\documentclass[12pt]{article}
\usepackage[utf8]{inputenc}

\begin{document}
\section{4 Phases of Problem Solving}
\subsection{Understanding the Problem}
To even begin solving the problem, you must fully understand
what the problem is suggesting. Identify the knowns and unknowns
within the problem. Ask yourself some of these questions before moving on.
Do you know what every word means in the problem? Are you able to explain
the problem to somebody? Do you know what you are trying to solve for? How do 
these variables given relate to what you are trying to solve? Did I miss anything?
Acting without fully understanding the problem first will often lead to failure.

\subsection{Devising a Plan}
Before diving into the problem solving, we must have an idea of 
how we are going to get to the solution. Diving head on first without a plan will likely result in failure when executing.
Think about how each variable relate to each other. Think of formulas, past problems, etc that relate or 
are similar to this problem. Do you have all the necessary tools to solve for what is being asked?
If you cannot come up with a plan on the problem, the problem is either too hard, you are missing information/concepts, or the problem is unsolvable 
with the given variables. Try devising a plan on a different variant of the problem if all else fails. 
The plan that you devise should be clear and concise. Each step of your plan should be correct.
There shouldn't be anything that you don't understand in your plan.

\subsection{Executing the Plan}
This one if fairly easy. You have already passed the hard part of problem solving.
The dangers of executing your plan can arise if you forgot a step or your step is wrong.
It is important that each step in your plan is correct and makes sense.

\subsection{Looking back}
Look back at your solution and reexamine it. Fully understand 
your solution. Are there things that you could've improved? Check
other solutions for the problem. What are somethings done differently?
There's always more room for improvement, you just have to look for it.
Looking back solidifies your knowledge on that problem which puts more tools into your problem solving toolbox.

\end{document}